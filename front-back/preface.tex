%!TEX root = ../terrainbook.tex

% About this book
% from a course, bundle of lecture notes
% link to videos
% acknowledgements


\chapter*{Preface}

This book presents an overview of algorithms and methodologies to reconstruct, manipulate, and extract information from terrains.

It covers different representations of terrains (\eg\ TINs, rasters, point clouds, contour lines), presents techniques to handle large datasets, and discusses related topics such as bathymetric datasets and global elevation models.

% DTM are often only grid and TINS
% Modelling of terrains is one aspect of GIS that significantly changed with the arrival of new acquisition technologies such as airborne laser scanners and radar (SRTM), and yet books are often written years ago.


\paragraph*{Open material.}
This book is the bundle of the lecture notes that we wrote for the course \emph{Digital terrain modelling} (GEO1015) in the MSc Geomatics at the Delft University of Technology in the Netherlands.
The course is tailored for MSc students who have already followed an introductory course in GIS and in programming.
Each chapter is a lesson in the course, whose content is also open: \url{https://3d.bk.tudelft.nl/courses/geo1015}.


\paragraph*{Accompanying videos.}
Most of the chapters have a short video explaining the key concepts, and those are freely available online: \url{https://tudelft3d.github.io/terrainbook}.


\paragraph*{Who is this book for?}
The book is written for students in Geomatics at the MSc level, but we believe it can be also used at the BSc level.
Prerequisites are: GIS, background in linear algebra, programming course at the introductory level.


\paragraph*{Acknowledgements.}
We thank Balázs Dukai for thoroughly proof-reading the drafts of this book, and the many students of the \href{https://3d.bk.tudelft.nl/courses/geo1015}{GEO1015 course} over the years who helped us by pointing--- and often fixing with a pull requests---the errors, typos, and weird sentences of this book. 
A special thank to the students of the year 2018--2019 who had to deal with the first version of this book.






