%!TEX root = ../terrainbook.tex
% chktex-file 46

\setchapterpreamble[u]{\margintoc}
\graphicspath{{gdem/figs/}}


\chapter{Global digital elevation models}% or global terrains
\label{chap:gdem}

We define as ``global digital elevation models'' (or global terrains) the datasets that cover (most of) the Earth (gDEM below).%
\index{global DEM}
Those datasets require different acquisition methods from the local ones, since flying an airplane or performing local surveys at the scale of the Earth is not really feasible (or is it?).
The acquisition instruments used must be space-borne, \ie\ mounted on a satellite for instance.

gDEMs are useful in several applications, especially for environmental studies such as geological studies, hydrological modelling, ecosystems dynamics.

gDEMS have several properties and characteristics that apply only to them, and we report in this chapter on the main ones.
% We first describe global acquisition techniques, then we discuss the properties (and errors and biases, etc.) that the datasets collected will have, 


%%%%%%%%%%%%%%%%%%%%
%
\section[Acquisition of global data]{Acquisition of global elevation data}

\begin{itemize}
  \item InSAR
  \item Photogrammetry from high-resolution satellite images
  \item Space lidar (ICESat-2 + GEDI)
\end{itemize}


%%%%%%%%%%%%%%%%%%%%
%
\section[Specific characteristics]{Specific characteristics of gDEMs}

\begin{itemize}
  \item CRS, especially vertical datums
  \item resolution (often in degrees)
  \item DSM (more than DTM)
  \item size of datasets
  \item errors
  \item integration with sea-level datasets
  \item accuracy affected by slope (most image-based products)
\end{itemize}


%%%%%%%%%%%%%%%%%%%%
%
\section[Most common products]{Most common products available}

We list only the ones available as open-access?
For instance AW3D is also available as 5m-grid, but €€€.

\begin{itemize}
  \item ASTER
  \item AW3D30
  \item STRM
  \item CopernicusDEM
  \item MERIT
  \item FABDEM
  \item ICESat-2 (the gridded version?)
  \item GEDI (the gridded version?)
\end{itemize}

A few words about fusion? [Okolie22] has very long review.


%%%%%%%%%%%%%%%%%%%%
%
\section{Conversion DSM to DTM}

Some examples of how done?

Or: correction bias for vegetations/trees and buildings (all man-made objects).

%%%%%%%%%%%%%%%%%%%%
%
\section{Notes \& comments}


%%%%%%%%%%%%%%%%%%%%
%
\section{Exercises}

\begin{enumerate}
  \item What is what?
\end{enumerate}
